\documentclass[a4paper]{article}

%% Language and font encodings
\usepackage[english]{babel}
\usepackage[utf8x]{inputenc}
\usepackage[T1]{fontenc}

%% Sets page size and margins
\usepackage[a4paper,top=3cm,bottom=2cm,left=3cm,right=3cm,marginparwidth=1.75cm]{geometry}

%% Useful packages
\usepackage{amsmath}
\usepackage{enumerate}
\setlength\parindent{0pt}
\usepackage{amssymb}
\setcounter{section}{-1}
\title{MA 1971 Exercise Set 3 Answers}
\author{Hubert J. Farnsworth}

\begin{document}
\maketitle

\begin{enumerate}

%%%%%%%%%%%%%%%%%%%%%%%%%%%%%%%%%%%%%%%%%%%%%%%%%%%%%%%%%%%
\item

Suppose that $p$ and $p+2$ are twin primes with $p > 3$. Suppose that $6 \nmid (p+1)$. Then one of the following cases must occur. In each case we will arrive at a contradiction.

If $p+1 = 6k+1$ then $p = 6k$. This shows that $6 \mid p$, which gives the contradiction that $p$ is not prime.\\

If $p+1 = 6k+2$ then $p+2 = 3(2k + 1)$. This shows that $3 \mid (p+2)$. Since $p+2 > 3$, $p+2$ must be a multiple of $3$ which gives the contradiction that $p+2$ is not prime.\\

If $p+1 = 6k+3$ then $p = 2(3k + 1)$. This shows that $2 \mid p$ Since $p > 3$, $p$ must be a multiple of $2$ which gives the contradiction that $p$ is not prime. \\

If $p+1 = 6k + 4$ then $p = 3(2k + 1)$. Similar to before, this implies the contradiction that $p$ is not prime. \\

If $p+1 = 6k + 5$ then $p+2 = 6(k+1)$. Similar to before, this implies the contradiction that $p+2$ is not prime.

Conclude that since $6 \nmid (p+1)$ implies a contradiction in any of the five possible cases given above, we must in fact have $6 \mid (p+1)$.

%%%%%%%%%%%%%%%%%%%%%%%%%%%%%%%%%%%%%%%%%%%%%%%%%%%%%%%%%%%
\item

\textbf{Lemma} For $a \in \mathbb{Z}$, $3 \mid a^2$ iff $3 \mid a$. \\

Proof: If $3 \mid a$ then $a = 3k$ for some integer $k$ and so $a^2 = 3(3k^2)$. Therefore $3 \mid a^2$. \\

Suppose $3 \mid a^2$ but $3 \nmid a$. Then either $a = 3k+1$ or $a = 3k+2$ for some integer $k$ from which we obtain $a^2 = 3(3k^2+2k)+1$ in the former case and $a^2 = 3(3k^2 + 4k + 1) + 1$ in the latter. In both cases we have contradicted the assumption that $3 \mid a^2$. Therefore, if $3 \mid a^2$ it must follow that $3 \mid a$.\\

Answer to Exercise 2: Suppose that $\sqrt{3}$ is rational. Then we can write $\sqrt{3} = p / q$ where $p\in \mathbb{Z}$, $q \in \mathbb{Z}^+$ and $\gcd(p,q) = 1$. Then, $3q^2 = p^2$ which shows that $3 \mid p^2$. But $3 \mid p^2$ implies that $3 \mid p$ so that we can write $p = 3k$ for some integer $k$. Then $3q^2 = p^2 = 9k^2$ and so $q^2 = 3k^2$. This shows that $3 \mid q^2$ and consequently that $3 \mid q$. Therefore $3 \mid p$ and $3 \mid q$, which contradicts the assumption that $\gcd(p,q) = 1$. 


%%%%%%%%%%%%%%%%%%%%%%%%%%%%%%%%%%%%%%%%%%%%%%%%%%%%%%%%%%%
\item

\begin{align*}
F_{n-1}^2 - 2(F_{n-2} - 1)^2
&= \left(2^{2^{n-1}} + 1\right)^2 - 2\left(2^{2^{n-2}} + 1 - 1\right)^2 \\
&= \left(2^{2^{n-1}}\right)^2 + 2\left(2^{2^{n-1}}\right) + 1 - 2\left(2^{2^{n-2}}\right)^2 \\
&= \left(2^{2^{n-1}}\right)^2 + 2\left(2^{2^{n-1}}\right) + 1 - 2\left(2^{2^{n-1}}\right) \\
&= \left(2^{2^{n-1}}\right)^2  + 1 \\
&= 2^{2^n} + 1 \\
&=: F_n
\end{align*}


%%%%%%%%%%%%%%%%%%%%%%%%%%%%%%%%%%%%%%%%%%%%%%%%%%%%%%%%%%%
\item

\textbf{Lemma} For $a \in \mathbb{Z}$, $2 \mid a^2$ iff $2 \mid a$.

Proof: The proof of this lemma is nearly identical (in fact slightly simpler) to the proof of the lemma used for Exercise 2 and is therefore omitted. 

Answer to Exercise 4: Let $x = 2m+1$ and $y = 2n+1$ are both positive integers with $m,n \in \mathbb{Z}$ and suppose that $x^2+y^2$ is a perfect square with $x^2 + y^2 = k^2$ for some $k \in \mathbb{Z}$.

\begin{align*}
k^2 &= x^2 + y^2 \\
&= (2m+1)^2 + (2n+1)^2 \\
&= 4m^2 + 4m + 1 + 4n^2 + 4n + 1 \\
&= 2(2m^2 + 2m + 2n^2 + 2n + 1) \\
&= 2[2(m^2 + m + n^2 + n) + 1] \\
&= 2r, \quad r:= 2(m^2 + m + n^2 +n) + 1
\end{align*}

This shows that $k^2 = 2r$ and that $r$ is an odd integer. Then $k^2$ is even, from which it follows that $k$ is even (by the lemma). Thus $k = 2q$ for some integer $q$ and $4q^2 = k^2 = 2r$. But then $2q^2 = r$, which produces the contradiction that $r$ is also even. Since this contradiction arose from supposing that $x^2 + y^2$ is a perfect square, conclude that if $x$ and $y$ are odd positive integers that $x^2 + y^2$ is not a perfect square. 


%%%%%%%%%%%%%%%%%%%%%%%%%%%%%%%%%%%%%%%%%%%%%%%%%%%%%%%%%%%	
\item

\textbf{Lemma} $3 \mid (10^k - 1) \quad \forall k \in \mathbb{Z}^+$. This is equivalent to the statement $10^k \equiv 1 \mod 3$.\\


Proof (Induction): \\
1) $k = 1$: It is clear that $3 \mid (10^1 - 1)$ since $10^1 - 1 = 9 = 3 \cdot 3$. \\
2) Suppose $3 \mid (10^k - 1)$ so that $10^k - 1 = 3j$ for some integer $j$. Then $10^{k+1} - 1 = 10(10^k - 1) + 9 = 10(3j) + 9 = 3(10j + 3)$ which shows that $3 \mid (10^{k+1} - 1)$.\\

Answer to Exercise 5: Let $n = d_0 + 10d_1 + 10^2d_2 + \dots + 10^kd_k$ for some $k \in \{0,1,2, \dots\}$ so that the digits of $n$ are $d_0, d_1,\dots , d_k \in \{0,1,\dots,9\}$. \\

$(\implies)$ Suppose that $3 \mid (d_0 + d_1 + \dots + d_k)$. Then there are integers $k, j_1, \dots , j_k$. 

\begin{align*}
3k &= d_0 + d_1 + \dots + d_k \\
3k + (10^1 - 1)d_1 &= d_0 + 10d_1 + d_2 + \dots + d_k \\
3k + (10^1 - 1)d_1 + (10^2 - 1)d_2 &= d_0 + 10d_1 + 10^2 d_2 + \dots + d_k\\
3k + (10^1 - 1)d_1 + (10^2 - 1)d_2 &= d_0 + 10d_1 + \dots 10^kd_k = n \\
3k + 3j_1d_1 + 3j_2d_2 + \dots + 3j_kd_k &= n \quad \text{(using the lemma)} \\
3(k + j_1d_1 + j_2d_2 + \dots + j_kd_k) &= n \implies 3 \mid n
\end{align*}

$(\impliedby)$ Suppose that $3 \mid n$. Then $3k = n = d_0 + 10d_1 + \dots + 10^kd_k$ for some integer $k$. Using the lemma, there exist integers $j_1, \dots , j_k$ such that: \\

\begin{align*}
3k &= d_0 + 10d_1 + 10^2d_2 + \dots + 10^kd_k \\
3k - (10^1 - 1)d_1  - (10^2 - 1)d_2 + \dots - (10^k-1)d_k
&= d_0 + d_1 + d_2 + \dots + d_k \\
3k - 3j_1d_1 - 3j_2d_2 - \dots - 3j_kd_k &= d_0 + d_1 + \dots + d_k \\
3(k - j_1d_1 - \dots - j_kd_k) &= d_0 + d_1 + \dots + d_k \implies 3 \mid (d_0 + \dots + d_k)
\end{align*}


$(\iff)$ A proof using modular arithmetic. This proof assumes a few basic properties of modular arithmetic. 
\begin{align*}
&3 \mid n \\
&\iff 0 \equiv n \mod 3 \\
&\iff 0 \equiv (d_0 + \dots 10^kd_k) \mod 3 \\
&\iff 0 \equiv d_0 \mod 3 + 10d_1 \mod 3 + \dots 10^kd_k \mod 3\\
&\iff 0 \equiv d_0 \mod 3 + d_1 \mod 3 + \dots + d_k \mod 3 \\
&\iff 0 \equiv (d_0 + d_1 + \dots + d_k) \mod 3 \\
&\iff 3 \mid (d_0 + d_1 + \dots + d_k)
\end{align*}

\end{enumerate}

\end{document}