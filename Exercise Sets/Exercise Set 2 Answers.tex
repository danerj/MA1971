\documentclass[a4paper]{article}

%% Language and font encodings
\usepackage[english]{babel}
\usepackage[utf8x]{inputenc}
\usepackage[T1]{fontenc}

%% Sets page size and margins
\usepackage[a4paper,top=3cm,bottom=2cm,left=3cm,right=3cm,marginparwidth=1.75cm]{geometry}

%% Useful packages
\usepackage{amsmath}
\usepackage{enumerate}
\setlength\parindent{0pt}
\usepackage{amssymb}
\setcounter{section}{-1}
\title{MA 1971 Exercise Set 2 Answers}
\author{Hubert J. Farnsworth}

\begin{document}
\maketitle

\begin{enumerate}

%%%%%%%%%%%%%%%%%%%%%%%%%%%%%%%%%%%%%%%%%%%%%%%%%%%%%%%%%%%
\item

Let $x \in (A \cap B)^c$. Then $x \not \in A\cap B$, which means $x \not \in A$ of $x \not \in B$. Equivalently, $x \in A^c$ or $B^c$. Thus $x \in A^c \cup B^c$. Therefore $ (A \cap B)^c \subseteq A^c \cup B^c$. Each of these steps is reversible. Therefore we have also $ (A \cap B)^c \supseteq A^c \cup B^c$.

	
%%%%%%%%%%%%%%%%%%%%%%%%%%%%%%%%%%%%%%%%%%%%%%%%%%%%%%%%%%%
\item

Let $x \in A$. Since $A \subset B$, $x \in B$. Since $B \subset C$, $x \in C$. Therefore $A \subset C$. 


%%%%%%%%%%%%%%%%%%%%%%%%%%%%%%%%%%%%%%%%%%%%%%%%%%%%%%%%%%%
\item

$$	
A_U^c = [0,3)\cup [7, 10], \quad
A_{\mathbb{R}}^c = (-\infty, 3) \cup [7, \infty), \quad
B_U^c = [0,3) \cup (3, 6) \cup (6, 9) \cup (9, 10] 
$$
	


%%%%%%%%%%%%%%%%%%%%%%%%%%%%%%%%%%%%%%%%%%%%%%%%%%%%%%%%%%%
\item

Consider the following counterexample. Let $A = \{a\}, B = \{b\}$ so that $A\cup B = \{a,b\}$.

\begin{align*}
\mathcal{P}(A) &= \{\emptyset, \{a\}\} \\
\mathcal{P}(B) & = \{\emptyset, \{b\} \} \\
\mathcal{P}(A) \cup \mathcal{P}(B) &= \{\emptyset, \{a\},\{b\} \} \\
\mathcal{P}(A\cup B) &= \{\emptyset, \{a\},\{b\}, \{a,b\} \} \\
\mathcal{P}(A) \cup \mathcal{P}(B) &\neq \mathcal{P}(A\cup B)
\end{align*}

%%%%%%%%%%%%%%%%%%%%%%%%%%%%%%%%%%%%%%%%%%%%%%%%%%%%%%%%%%%	
\item

\begin{enumerate}

	\item
	
	Base case: $\frac{1(1+1)}{2} = 1 = \sum_{i=1}^1 i$.\\
	Inductive step: Suppose for $k \in \mathbb{Z}^+$ that $			\sum_{i = 1}^k i = \frac{k(k+1)}{2}$. Then,
	
	$$
	\sum_{i=1}^{k+1} i
	= k+1 + \sum_{i=1}^k i
	= k+1 + \frac{k(k+1)}{2}
	= \frac{2(k+1) + k(k+1)}{2}
	= \frac{(k+1)((k+1)+1}{2} \;.
	$$
	
	The claim holds by the principle of induction.
	
	\item
	
	Base case: $\frac{1(1+1)(2+1)}{6}=1=1^2=\sum_{i=1}^1i^2$.\\
	Inductive step: Suppose for $k \in \mathbb{Z}^+$ that $			\sum_{i = 1}^k i^2 = \frac{k(k+1)(2k+1)}{6}$. Then,
	
	\begin{align*}
	\sum_{i=1}^{k+1} i^2 &= (k+1)^2 + \sum_{i=1}^k i^2
	= (k+1)^2 + \frac{k(k+1)(2k+1)}{6} \\
	&= \frac{6(k+1)^2 + k(k+1)(2k+1)}{6} \\
	&= \frac{(k+1)[6(k+1) + 2k^2 + k]}{6} \\
	&= \frac{(k+1)[2k^2 + 7k + 6]}{6} \\
	&= \frac{(k+1)[2k(k + 2) + 3(k + 2)]}{6} \\
	&= \frac{(k+1)(k+2)(2k + 3)}{6} \\
	&= \frac{(k+1)((k+1)+1)(2(k+1) + 1)}{6}\ \\
	\end{align*}
	
	The claim holds by the principle of induction.

\end{enumerate}

	
%%%%%%%%%%%%%%%%%%%%%%%%%%%%%%%%%%%%%%%%%%%%%%%%%%%%%%%%%%%
\item

\begin{enumerate}

	\item
	{\bf Claim}: For any positive integer $n$,
	$6 \mid (n^3 - n)$. \\
	Proof (Induction with two base cases): \\
	Base case: $6 \mid (1^3 - 1)$ since $1^3 - 1 = 0$ and
	$6 \mid 0$. Also, $6 \mid (2^3 - 2)$ since $2^3 - 2 = 6$
	and $6 = 6 \cdot 1 \implies 6 \mid 1$.\\
	Inductive step: Suppose for some $k \in \mathbb{Z}^+$ that
	$6 \mid k^3 - k$. Then $k^3 - k = 6j$ for some integer $j$.
	Consider the case for $k+2$. 
	$$
	(k+2)^3 - (k+2) = k^3 + 6k^2 + 12k + 8 - (k+2)
	= k^3 - k +6k^2 + 12k + 6
	= 6j + 6(k^2 + 2k + 1)\;.
	$$
	Thus $(k+2)^3 - (k+2) = 6(j + k^2 + 2k + 1)$, which shows
	that $6 \mid [(k+2)^3 - (k+2)]$. 
	
	\item
	{\bf Lemma}
	For any $n \in \mathbb{Z}^+$, $6 \mid 3n(n+1)$. \\
	Proof (Induction): \\
	Base case: For $6 \mid 3(1)(1+1)$ since $3(1)(1+1) = 6$. \\
	Inductive step: Suppose for some $k \in \mathbb{Z}^+$ that
	$6 \mid 3k(k+1)$. Then $3k(k+1) = 3k^2 + 3k = 6j$ for some
	integer $j$. Consider the case for $k + 1$.
	$$
	3(k+1)(k+2) = 3k^2 + 9k + 6
	= 3k^2 + 3k + 6k + 6
	=6(j + k + 1) \;.
	$$
	The lemma holds by the principle of induction.\\
	
	{\bf Claim}: For any positive integer $n$,
	$6 \mid (n^3 - n)$.\\
	Proof (Induction): 
	Base case: $6 \mid (1^3 - 1)$ since $1^3 - 1 = 0$ and 
	$6 \mid 0$. \\
	Inductive step: Suppose for some $k \in \mathbb{Z}^+$ that
	$6 \mid (k^3 - k)$.
	Then $k^3 - k = 6j$ for some integer $j$.
	Consider the case for $k+1$. 
	$$
	(k+1)^3 - (k+1) = k^3 + 3k^2 + 3k + 1 - (k+1)
	= k^3 - k + 3k^2 + 3k
	= 6j + 6l
	= 6(j+l) \; ,
	$$
	where we used the inductive hypothesis to get 
	$k^3 - k = 6j$ and the lemma to get $3k^2 + 3k = 6l$ for 
	some integer $l$. 
	This shows that $6 \mid [(k+1)^3 - (k+1)]$ and so the
	claim holds by the principle of induction. 

\end{enumerate}

	
%%%%%%%%%%%%%%%%%%%%%%%%%%%%%%%%%%%%%%%%%%%%%%%%%%%%%%%%%%%
\item

Let $A$ and $B$ be sets such that $A\subset B$ and $Q(x)$ be the predicate $Q(x) = x \not\in A$. By Axiom III (specification), there exists a set $S$ such that
$$
S = \{x \in B \mid Q(x) \text{ is true} \}
= \{x \in B \mid x \not \in A\}
=: A_B^c \;.
$$

	

%%%%%%%%%%%%%%%%%%%%%%%%%%%%%%%%%%%%%%%%%%%%%%%%%%%%%%%%%%%
\item

Proof: Suppose there exists a set $S$ such that $S$ contains all sets. 
We have seen that $|\mathcal{P}(S)| = 2^{|S|} > |S|$. But since $\mathcal{P}(S)$ is also a set, this would imply that $|\mathcal{P}(S)| \leq |S|$. Thus we have the contradiction,
$|\mathcal{P}(S)| < |\mathcal{P}(S)|$. \\

Proof: Suppose there exists a set $S$ that contains all sets. By Axiom III, there exists a set $B = \{s \in S \mid s \not\in s\}$. Since $S$ contains all sets, $B \in S$. If $B \in B$, then by definition of $B$, $B \not \in B$. But if $B \not \in B$, then since $B \in S$ and $B \not \in B$, we have $B \in B$. Therefore, if there exists a set $S$ containing all sets, there must exists a set $B$ such that $B \in B$ and $B \not \in B$. Since $B \in B$ and $B \not \in B$ is impossible, there cannot exist a set $S$ containing all sets.  

%%%%%%%%%%%%%%%%%%%%%%%%%%%%%%%%%%%%%%%%%%%%%%%%%%%%%%%%%%%
\item

No, $\emptyset \neq \{\emptyset\}$. By Axiom II, two sets are equal iff they contain the same elements. However, $|\emptyset| = 0$ while $|\{\emptyset \}| = 1$. Since $\emptyset$ and $\{\emptyset \}$ do not even contain the same number of elements, it is impossible for these two sets to contain the same elements. 
%%%%%%%%%%%%%%%%%%%%%%%%%%%%%%%%%%%%%%%%%%%%%%%%%%%%%%%%%%%
\item

In lecture, we used Axioms I,II, and III to prove that there exists uniquely the empty set $\emptyset$. Taking $A = B = \emptyset$ in Axiom IV (pairing), there exists a set $\mathcal{C}$ such that $\emptyset \in \mathcal{C}$. Using Axiom III, we obtain the existence of the set $\{\emptyset \} = \{X \in \mathcal{C} \mid X = \emptyset \}$. Thus we have proved the existence of the sets $\emptyset$ and $\{\emptyset \}$. Taking $A = \emptyset$ and $B = \{ \emptyset \}$ in Axiom IV, there exists a set $\mathcal{C}$ such that $\emptyset, \{\emptyset \} \in \mathcal{C}$. Using Axiom III, we obtain $\{\emptyset, \{\emptyset \} \} = \{X \in \mathcal{C} \mid X = \emptyset \lor X = \{\emptyset \} \}$. Taking $A = B = \{\emptyset \}$ in Axiom IV, there exists a set $\mathcal{C}$ such that $\{\emptyset \} \in \mathcal{C}$. Then by Axiom III there exists the set $\{\{\emptyset \} \} = \{ X \in \mathcal{C} \mid X = \{\emptyset \} \}$. Then similarly, by applying Axiom IV and III  as we just did but with $A = B = \{ \{\emptyset \} \}$ we obtain the existence of the set $\{\{\{\emptyset \} \} \}$. \\

Take $A = \{\emptyset , \{\emptyset\} \}$ and $B = \{\{\{\emptyset \}\}\}$. By Axiom IV there exists a set $\mathcal{C}$ such that $\{\emptyset, \{\emptyset\} \} \in \mathcal{C}$ and $\{\{\{\emptyset \}\}\} \in \mathcal{C}$. By Axiom III, there exists a set
$$
\mathcal{A} = \{\; \{\emptyset, \{\emptyset\} \}, \{\{\{\emptyset\}\}\} \; \} = \{ \; X \in \mathcal{C} \mid X = \{\emptyset, \{\emptyset\} \} \lor X = \{\{\{\emptyset\}\}\} \} \; \}
$$

By Axiom V (unioning), there exists a set $\cup \mathcal{A}$ such that,

$$
\cup \mathcal{A}
= \{x \in A \mid A \in \mathcal{A} \}
= \{\;\emptyset, \{\emptyset \}, \{\{\emptyset \}\} \;\} \;.
$$



	
\end{enumerate}


\end{document}